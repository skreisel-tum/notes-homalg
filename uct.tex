\section{The Universal Coefficient Theorem}

In this section we work exclusively with $\mathbb{Z}$-modules, i.e. abelian groups, and drop the ring sup/subscripts.

\textbf{Definition. (Co)Homology with Coefficients.} \ Let $A$ be an abelian group and $C_\bullet$ a chain complex of free abelian groups, i.e. free $\mathbb{Z}$-modules. We define
\[
  H_n(C_\bullet,A) ~:=~H_n(C_\bullet \otimes A)\quad \text{and}\quad H^n(C_\bullet, A) ~:=~H^n\big(\text{Hom}(C_\bullet, A)\big)
\]
to be the homology and cohomology with coefficents in $A$ respectively.

\textbf{Theorem. Universal Coefficient Theorem.} \ For general right exact functors. Let $\mathcal{A}$ be an abelian category that has enough projectives and let $F: \mathcal{A} \to \mathcal{B}$ be right exact. Further, let $(C_\bullet,d_\bullet)$ be a chain complex in $\text{Ch}(\mathcal{A})$ such that $C_n$ and $\text{im}(d_n)$ are $F$-acyclic for all $n \in \mathbb{N}_0$ (or more generally $n \in \mathbb{Z}$).\\
  Then we have a natural short exact sequence
  \[
  \begin{tikzcd}[column sep=2.5em, row sep=2.5em]
    0 \rar & F\big(H_n(C_\bullet)\big) \rar & H_n\big(F(C_\bullet)\big) \rar & L_1\big(F(H_{n-1}(C_\bullet))\big) \rar & 0
  \end{tikzcd}
    \]
    for all $n \geq 1$. If all $\text{im}(d_n)$ are projective, then this short exact sequence splits, although not canonically.

    \textbf{Theorem. Universal Coefficient Theorem.} \ Let $A \in \textsc{Ab}$ and $C_\bullet$ be a chain complex of free abelian groups. Then we have natural short exact sequences
    \[
  \begin{tikzcd}[column sep=2em, row sep=2em]
    0 \rar & H_n(C_\bullet) \otimes A \rar & H_n((C_\bullet \otimes A) \rar & \text{Tor}_1\big(H_{n-1}(C_\bullet), A\big)\rar & 0
  \end{tikzcd}
      \]
      and
      \[
  \begin{tikzcd}[column sep=2em, row sep=2em]
    0 \rar & \text{Ext}^1\big(H_{n-1}(C_\bullet, A)\big)\rar & H^n\big( \text{Hom}(C_\bullet, A) \big) \rar & \text{Hom}\big( H_n(C_\bullet), A \big)\rar & 0
  \end{tikzcd}
      \]
      that split non-canonically. In particular we have isomorphisms
      \[
 H_n(C_\bullet, A) ~~\cong~~ \big(H_n(C_\bullet) \otimes A\big) ~\oplus~ \text{Tor}_1\big(H_{n-1}(C_\bullet), A\big)
      \]
      and
      \[
  H^n(C_\bullet, A) ~~\cong~~\text{Hom}\big(H_n(C_\bullet), A) ~\oplus~ \text{Ext}^1\big(H_{n-1}(C_\bullet), A)\big).
      \]

      \textbf{Proof.} \ This follows from the general version of the theorem with $F = (-\otimes A): \textsc{Ab} \to \textsc{Ab}$ and $F = \text{Hom}(-,A): \textsc{Ab} \to \textsc{Ab}^{\text{op}}$. In particular we assume $C_\bullet$ to be free and thus projective and thus $F$-acyclic. Also $\text{im}(d_n)$ is free again as a subgroup of a free group.

      \textbf{Proof of the UCT.} \ We are given a right exact functor $F: \mathcal{A} \to \mathcal{B}$ and a chain complex $C_\bullet$ that is $F$-acyclic in all degrees. The strategy to proof the UCT is as follows
      \begin{enumerate}
        \item Construct a short exact sequence of chain complexes
          \[
          \begin{tikzcd}[column sep=2.5em, row sep=2.5em]
            0 \rar &
            Z_\bullet \arrow[hook]{r}{} &
            C_\bullet \arrow[two heads]{r}{} &
            B_{\bullet - 1} \rar &
            0
          \end{tikzcd}
          \]
          where $Z_\bullet$ and $B_\bullet$ have $d \equiv 0$.
        \item Apply $F$ to this sequence to obtain a short exact sequence
          \[
          \begin{tikzcd}[column sep=2.5em, row sep=2.5em]
            0 \rar &
            FZ_\bullet \arrow[hook]{r}{} &
            FC_\bullet \arrow[two heads]{r}{} &
            FB_{\bullet - 1} \rar &
            0
          \end{tikzcd}
          \]
          in $\text{Ch}(\mathcal{B})$.
        \item Apply the homology functor $H_n$ to obtain the long exact sequence for homology
  \[
  \begin{tikzcd}[column sep=2.5em, row sep=2.5em]
    &
    &
    ~~\cdots~~\rar \ar[draw=none]{d}[name=X, anchor=center]{} &
    H_{n+1}(FB_{\bullet -1}) \ar[rounded corners, to path={ -- ([xshift=2ex]\tikztostart.east)
                                        |- (X.center) \tikztonodes
                                        -| ([xshift=-2ex]\tikztotarget.west)
                                        -- (\tikztotarget)}]{dll}[at end]{\delta_{n+1}}
  \\
  &
    H_n(FZ_\bullet) \rar &
    H_n(FC_\bullet)\rar \ar[draw=none]{d}[name=Y, anchor=center]{} &
    H_n(FB_{\bullet - 1}) \ar[rounded corners, to path={ -- ([xshift=2ex]\tikztostart.east)
                                        |- (Y.center) \tikztonodes
                                        -| ([xshift=-2ex]\tikztotarget.west)
                                        -- (\tikztotarget)}]{dll}[at end]{\delta_n}
  \\
  &
    H_{n-1}(FZ_\bullet) \rar &
    ~~\cdots &
  \end{tikzcd}
    \]
  \item Since $Z_\bullet$ and $B_\bullet$ have zero differential (which is still the case after applying $F$) we have
  \[
  \begin{tikzcd}[column sep=2.5em, row sep=2.5em]
    &
   &
    ~~\cdots~~\rar \ar[draw=none]{d}[name=X, anchor=center]{} &
    FB_n\ar[rounded corners, to path={ -- ([xshift=2ex]\tikztostart.east)
                                        |- (X.center) \tikztonodes
                                        -| ([xshift=-2ex]\tikztotarget.west)
                                        -- (\tikztotarget)}]{dll}[at end]{\delta_{n+1}}
  \\
  &
    FZ_n \rar &
    H_n(FC_\bullet)\rar \ar[draw=none]{d}[name=Y, anchor=center]{} &
    FB_{n-1}\ar[rounded corners, to path={ -- ([xshift=2ex]\tikztostart.east)
                                        |- (Y.center) \tikztonodes
                                        -| ([xshift=-2ex]\tikztotarget.west)
                                        -- (\tikztotarget)}]{dll}[at end]{\delta_{n}}
  \\
  &
    FZ_{n-1}\rar &
    ~~\cdots &
  \end{tikzcd}
    \]
  \item From the long exact sequence we obtain a short exact sequence
    \[
          \begin{tikzcd}[column sep=2.5em, row sep=2.5em]
            0 \rar &
            \text{coker}(\delta_{n+1}) \rar &
            H_n(FC_\bullet) \rar &
            \text{ker}(\delta_n) \rar &
            0
          \end{tikzcd}
    \]
  \item We show that $\delta = F(i)$.
  \item We have a short exact sequence
    \[
          \begin{tikzcd}[column sep=2.5em, row sep=2.5em]
            0 \rar &
            B_n \arrow[hook]{r}{i} &
            Z_n \arrow[two heads]{r} &
            H_n(C_\bullet) \rar &
            0
          \end{tikzcd}
    \]
    which is a $F$-acyclic resolution of $H_n(C_\bullet)$. Applying $F$ yields the long exact sequence for the left derived functor
    \[
          \begin{tikzcd}[column sep=2.5em, row sep=2.5em]
   &
    ~~\cdots~~\rar \ar[draw=none]{d}[name=X, anchor=center]{} &
            L_1(F)(H_{n}C_\bullet)\ar[rounded corners, to path={ -- ([xshift=2ex]\tikztostart.east)
                                        |- (X.center) \tikztonodes
                                        -| ([xshift=-2ex]\tikztotarget.west)
                                        -- (\tikztotarget)}]{dll}[at end]{}
                                        &
  \\
            FB_{n-1} \arrow{r}{Fi} &
    FZ_{n-1} \rar &
    FH_n(C_\bullet) \rar &
            0
          \end{tikzcd}
    \]
  \item But then
    \[
    \text{coker}(Fi) ~\cong~ H_n(FC_\bullet),\qquad \text{ker}(Fi) ~\cong~ L_1(F)(H_nC_\bullet)
    \]
  \item For the kernel we have to shift one degree down and substituting in (5) yields
    \[
  \begin{tikzcd}[column sep=2.5em, row sep=2.5em]
    0 \rar & F(H_n(C_\bullet) \rar & H_n(FC_\bullet) \rar & L_1(F)(H_{n-1}(C_\bullet)) \rar & 0
  \end{tikzcd}
    \]
    as desired.
      \end{enumerate}
