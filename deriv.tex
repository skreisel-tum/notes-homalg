\section{Derived Functors}

In this section we work with abelian categories.

We have seen that $\text{Hom}(-,-): \mathcal{A}^\text{op}\times\mathcal{A} \to \textsc{Ab}$ is not a exact functor but only a left exact functor, failing to preserve short exact sequences at the right end. In this section we develop a tool to quanitfy by how much it fails to preserve exactness at the right when applied to a short exact sequence.

Generally, from a left exact functor $F: \mathcal{A} \to \mathcal{B}$ we derive a sequence of new functors $R^i(F): \mathcal{A} \to \mathcal{B}$ such that $R^0(F) = F$ and for any short exact sequence in $\mathcal{A}$
\[
  0 \longrightarrow A \longrightarrow B \longrightarrow C \longrightarrow 0
  \]
we not only obtain a exact sequence
\[
  0 \longrightarrow F(A) \longrightarrow F(B) \longrightarrow F(C)
  \]
  but also a long exact sequence
  \[
  \begin{tikzcd}[column sep=2.5em, row sep=2.5em]
    0 \rar &
    F(A) \rar &
    F(B) \rar \ar[draw=none]{d}[name=X, anchor=center]{} &
    F(C) \ar[rounded corners, to path={ -- ([xshift=2ex]\tikztostart.east)
                                        |- (X.center) \tikztonodes
                                        -| ([xshift=-2ex]\tikztotarget.west)
                                        -- (\tikztotarget)}]{dll}[at end]{\delta}
  \\
  &
  R^1(F)(A) \rar &
  R^1(F)(B) \rar \ar[draw=none]{d}[name=Y, anchor=center]{} &
  R^1(F)(C) \ar[rounded corners, to path={ -- ([xshift=2ex]\tikztostart.east)
                                        |- (Y.center) \tikztonodes
                                        -| ([xshift=-2ex]\tikztotarget.west)
                                        -- (\tikztotarget)}]{dll}[at end]{\delta}
  \\
  &
  R^2(F)(A) \rar &
  R^2(F)(B) \rar &
  ~~\cdots
  \end{tikzcd}
    \]
with connecting morphisms $\delta$ where $R^1(F)(A)$ (and the tail of the sequence) is a measure for the extent that $F$ fails to be (right) exact.

The sequence $R^i(F)$ for $i \geq 0$ together with the connecting map $\delta$ is called a cohomological delta functor.\\
We then show that the way we constructed this cohomological delta functor, the derived functor of $F$, is not arbitrary but really `the' derived functor. It is a universal cohomological delta functor.\\
Universal cohomological delta functors are those that determine natural transformations for delta functors completely by the 0th degree. But any derived functor of $F$, in its 0th degree, is naturally isomorphic to $F$ by assumption.
%Let $T^i, \delta$ be a cohomological delta functor. Given a natural transformations $\alpha^i: T^i \to S^i$ of cohomological delta functors that are compatible with their respective connecting maps $\delta$ we call $T^i$ universal if $\alpha^i$ is completely determined by the values of $\alpha^0: T^0 \to S^0$.
