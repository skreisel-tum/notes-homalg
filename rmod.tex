\section{Abelian Groups and Modules}

\textbf{Note. Module.} \ Let $R$ be a ring with 1. We define the category of left $R$-modules $R$-$\textsc{Mod}$ to be abelian groups (written additively) with an additional left action of ring elements in $R$, written $r \cdot m$ such that
\begin{enumerate}
  \item $r\cdot (x + y) = r \cdot x + r \cdot y$,
  \item $(r + s)\cdot x = r \cdot x + s \cdot y$,
  \item $(rs) \cdot x = r\cdot (sx)$,
  \item $1\cdot x = x$.
\end{enumerate}
Similarly we define $\textsc{Mod}$-$R$ of right $R$-modules. If $R$ is commutative then there is not difference between left and right modules.

\textbf{Note. Abelian Groups as Modules.} \ Any $\mathbb{Z}$-module merely carries the structure of an abelian group since the above axioms force $r \cdot m = m + m + \cdots + m$.

\textbf{Note. $R$ as a $R$-module.} \ Any ring $R$ can be made into a module by setting the action of $R$ on the abelian group of $R$ to be the ring multiplication, i.e. $r \cdot s = rs$.

\textbf{Note. Free Module.} \ An $R$-module $M$ is called free if it admits a basis, i.e. any $m$ can be written as a sum $m = r_1b_1 + \ldots + r_nb_n$ for elements $r_i \in R$ and $b_i \in M$ (may not be a finite sum).\\
Equivalently, free modules are isomorphic to the direct product of copies of $R$ as $R$-modules.

\textbf{Definition. $R$-Balanced Map.} \ Let $M \in \textsc{Mod-R}$ and $N \in \textsc{R-Mod}$ and $A \in \textsc{Ab}$. A map $\beta: M \times N \to A$ is called $R$-balanced if
\begin{enumerate}
  \item $\beta(a+b,x) = \beta(a,x) + \beta(b,x)$,
  \item $\beta(a,x+y) = \beta(a,x) + \beta(a,y)$,
  \item $\beta(ar,x) = \beta(a,rx)$.
\end{enumerate}

\textbf{Definition. Tensor Product.} \ Let $M \in \textsc{Mod-R}$ and $N \in \textsc{R-Mod}$. The tensor product of $M$ and $N$ over $R$ is an abelian group denoted $M \otimes_R N$ together with an $R$-balanced map denote $-\otimes-: M \times N \to M \otimes_R N$ such that for any $R$-balanced map $\beta: M \times N \to A$ to some abelian group $A$ there exists a unique group homomorphism $M \otimes_R N \to A$ such that the following diagram commutes:
\[
  \begin{tikzcd}[column sep=2.5em, row sep=3.5em]
    M \times N \arrow{rr}{-\otimes_R -} \arrow[swap]{dr}{\beta} & & M \otimes_R N \arrow[dashed]{dl}{\exists} \\ & A &
  \end{tikzcd}
  \]

  \textbf{Proposition. The Tensor Product Exists.} \ TODO

  \textbf{Proposition. The Tensor Product is a Bifunctor.} \ 
