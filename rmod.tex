\section{Abelian Groups and Modules}

\textbf{Note. R-Module.} \ Let $R$ be a ring with 1. We define the category of left $R$-modules $R$-$\textsc{Mod}$ to be abelian groups (written additively) with an additional left action of ring elements in $R$, written $r \cdot m$ such that
\begin{enumerate}
  \item $r\cdot (x + y) = r \cdot x + r \cdot y$,
  \item $(r + s)\cdot x = r \cdot x + s \cdot y$,
  \item $(rs) \cdot x = r\cdot (sx)$,
  \item $1\cdot x = x$.
\end{enumerate}
Similarly we define $\textsc{Mod}$-$R$ of right $R$-modules. If $R$ is commutative then there is not difference between left and right modules.

\textbf{Note. Abelian Groups as Modules.} \ Any $\mathbb{Z}$-module merely carries the structure of an abelian group since the above axioms force $r \cdot m = m + m + \cdots + m$.

\textbf{Note. $R$ as a $R$-module.} \ Any ring $R$ can be made into a module by setting the action of $R$ on the abelian group of $R$ to be the ring multiplication, i.e. $r \cdot s = rs$.

\textbf{Note. Free Module.} \ An $R$-module $M$ is called free if it admits a basis, i.e. any $m$ can be written as a sum $m = r_1b_1 + \ldots + r_nb_n$ for elements $r_i \in R$ and $b_i \in M$ (may not be a finite sum).\\
Equivalently, free modules are isomorphic to the direct product of copies of $R$ as $R$-modules.

\textbf{Proposition. Hom-Set as R-Module.} \ Let $R$ be commutative and $M, N$ be $R$-modules. Then the set $\text{Hom}(N,M)$ carries the structure of a $R$-module by
\[
  (r\cdot f)(m) = f(r\cdot m)
\]
for all $r \in R$, $m \in M$ and $f: M \to N$. In particular the functor $\text{Hom}(-,-): R\textsc{-Mod} \to \text{Ab}$ now lands in $R$-\textsc{Mod}. We write $\text{Hom}_R(M,N)$ to emphasize the module structure.
