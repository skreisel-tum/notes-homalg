\section{Group Homology}

\textbf{Note. G-Module.} \ Let $G$ be a group. We define the category of left $G$-modules \textsc{G-Mod} to be abelian groups (written additively) with an additional left action of group elements in $G$, written $g\cdot m$ such that
\begin{enumerate}
  \item $g\cdot (x + y) = g \cdot x + g \cdot y$,
  \item $(gh) \cdot x = g\cdot (hx)$,
  \item $1\cdot x = x$.
\end{enumerate}

$G$-modules are similar to $R$-modules, but lack the multiplication of ring elements. A $G$-module homomorphism $f: M \to N$ is a group homomorphism of abelian groups that satisfy $f(g\cdot a) = g\cdot f(a)$ (one could say it is $G$-linear).

\textbf{Note.} \ \textsc{G-Mod} is an abelian category.

\textbf{Note.} \ Fix a group $G$. Denote by $BG$ the category with one element and morphisms $* \to *$ equal to elements in $G$. Then
\[
  \text{Fun}(BG,\textsc{Ab}) ~\cong~\textsc{G-Mod} \quad \text{by}\quad F \mapsto F(*)
\]
where $g$ facts on $F(*)$ by the group homomorphism $F(g): F(*) \to F(*)$, i.e. $g\cdot m = F(g)(m) \in F(*)$ for all $m \in F(*)$ and $g \in G$.


\textbf{Note. Trivial G-Module.} \ Let $G$ be a fixed group and $M$ be an abelian group. Then there is a distinguished $G$-module structure on $M$ by $g\cdot m = m$ for all $g \in G$. This is called the trivial $G$-module of $M$, denoted by $\text{triv}(M)$ and $\underline{M}$.\\
In fact for a fixed group $G$ the map $\text{triv}(-): \textsc{Ab} \to \textsc{G-Mod}$ is a exact and fully faithful functor. If $f: M \to N$ is a morphism of abelian groups then clearly $\text{triv}(f) = f$ respects the (trivial) scalar multiplication with elements from $G$.

\textbf{Definition. Integral Group Ring.} \ Let $G$ be a group. We can form a free $\mathbb{Z}$-module by
\[
  \mathbb{Z}G ~:=~\bigoplus_{g\in G}\mathbb{Z}g.
\]
This module also carries the structure of a ring with multiplication on a basis of $\mathbb{Z}G$ given by the group law of $G$, i.e. $g_ig_j = g_i\cdot g_j$ that extends $\mathbb{Z}$-linearly.

\textbf{Example.} \ Let $G = \{ 1, a, a^2 \}$ be the cyclic group of three elements, multiplication of the ring $\mathbb{Z}G$ of elements $r = z_01 + z_1a + z_2a^2$ and $s = w_01 + w_1a + w_2a^2$ is given by
\[
  rs ~=~ z_0w_01 + z_0w_1a + z_0w_2a^2 + \cdots
\]
where $z_iw_i$ denotes multiplication in $\mathbb{Z}$.

\textbf{Definition. Augmentation Map.} \ Let $G$ be a group. The canonical ring homomorphism
\[
  \varepsilon: \mathbb{Z}G \to \mathbb{Z},\qquad g \mapsto 1
\]
is called augmentation map and sum all coefficients of any element $m \in \mathbb{Z}G$.

\textbf{Example.} \ Let $G = C_n$ be the cyclic groups on $n$ elements. Then $\mathbb{Z}G \cong \mathbb{Z}[t]/(t^n - 1)$, the ring of polynomials of finite degree at most $n - 1$.

\textbf{Example.} \ Let $G = C_\infty$ be the infinite cycle group. Then $\mathbb{Z}G \cong \mathbb{Z}[t^{-1},t]$, the ring of Laurent polynomials.

\textbf{Proposition. G-Modules as Ring Modules.} \ Let $G$ be a fixed group and $M$ a $G$-module. Then we can view $M$ as a $R$-module (ring module) for $R = \mathbb{Z}G$ by
\[
  g\cdot m ~\mapsto~ (1g) \cdot m,\qquad \Bigg(\sum_{i = 1}^{|G|}z_ig_i\Bigg) \cdot m ~\mapsto~ \sum z_i(g_i\cdot m).
\]
This is an isomorphism of categories $G$-\textsc{Mod} and $\mathbb{Z}G$-\textsc{Mod}.

\textbf{Definition. Group of (Co)Invariants.} \ Let $G$ be a fixed group and $M$ a $G$-module. We define abelian groups
\[
  M^G = \big\{ m \in M ~|~ g\cdot m = m\quad \forall g \in G\big\}
\]
and
\[
 M_G = M/(g\cdot m - m ~|~g \in G,~m\in M)
\]
called group of invariants and group of coinvariants respectively. $M^G$ is a subgroup of $M$ and $M_G$ is a quotient of $M$, both discarding the $G$-module structure.\\
Note that a $G$-module morphism $f: M \to N$ descends to morphisms $f^G: M^G \to N^G$ and $f_G: M_G \to N_G$. The former is well defined since for any $a \in M^G$ and $g \in G$ we have
\[
  g\cdot f(a) = f(g\cdot a) = f(a) \in N^G.
\]
Thus $(-)^G$ and $(-)_G$ are functors $G$-\textsc{Mod} to $\textsc{Ab}$.

\textbf{Proposition. Invariants and Adjoints.} \ The functors $\text{triv}(-)$, $(-)^G$ and $(-)_G$ are adjoint as follows
\[
  \begin{tikzcd}[column sep=5.5em, row sep=8.5em]
    G-\textsc{Mod} \arrow[bend right=70, swap]{d}{(-)_G} \arrow[bend left=70]{d}{(-)^G} \\
    \textsc{Ab} \arrow{u}{\text{triv}(-)}
  \end{tikzcd}
\]
such that $(-)_G \dashv \text{triv}(-)$ and $\text{triv}(-) \dashv (-)^G$, i.e. coinvariants are left-adjoint to the triv and invariants are right adjoint to triv.

\textbf{Proposition.} \ The invariant functor $(-)^G$ is additive and left exact, the coinvariant functor $(-)_G$ is additive and right exact.

\textbf{Note.} \ For a fixed group $G$ the category $G$-\textsc{Mod} has enough projectives and injectives since it is isomorphic to $\mathbb{Z}G$-\textsc{Mod}.

\textbf{Definition. Group (Co)Homology.} \ Let $G$ be a fixed group. We define the functors
\[
  H^n(G,-) ~:=~R^n(-)^G,\qquad H_n(G,-) ~:=~L_n(-)_G.
\]

\textbf{Proposition. Reinterpreting G-Modules.} \ We have natural isomorphisms
\[
  (-)_G ~\cong~\mathbb{Z}~\otimes_{\mathbb{Z}G}~-,\qquad (-)^G ~\cong~\text{Hom}_{\mathbb{Z}G-\textsc{Mod}}(\mathbb{Z},-)
\]
where $\mathbb{Z}$ appears as the (trivial) $\mathbb{Z}G$-module.

\textbf{Proposition. Reinterpreting Group (Co)Homology.} \ We have isomorphisms
\[
  R^n(-)^G ~=~H^n(G,-) ~\cong~\text{Ext}_{\mathbb{Z}G}^n(\mathbb{Z},-)
\]
and
\[
  L_n(-)_G ~=~ H_n(G,-) ~\cong~\text{Tor}^{\mathbb{Z}G}_n(\mathbb{Z},-)
\]
where $\mathbb{Z}$ is the tirival $\mathbb{Z}G$-module.

To compute Ext and Tor we can thus find a projective resolution of $\mathbb{Z}$ as a $\mathbb{Z}G$-module independent of the input in the second variable.

\textbf{Proposition. The Bar Resolution.} \ Let $n \geq 0$ and let $G$ be a fixed group. We define the free $\mathbb{Z}G$-module
\[
  B_n ~:=~ \bigoplus_{(g_i) \in (G - \{1 \})^n}\mathbb{Z}G[g_1|\ldots |g_n]
\]
where $[g_1|\ldots |g_n]$ denotes the generator corresponding to the $n$-tuple $(g_i)$. We further define
\[
  \begin{aligned}
    d_n: B_n ~\to~ B_{n-1},\qquad [g_1|\ldots |g_n] ~\mapsto~ &g_1[g_2|\ldots|g_n]\\
    & - [g_1g_2|\ldots |g_n]\\
    & \pm ~\cdots\\
    & + (-1)^{n-1} [g_1|\ldots|g_{n-1}g_n] \\
    & + (-1)^n [g_1|\ldots|g_{n-1}].
  \end{aligned}
\]
Then $(B_\bullet,d)$ is a chain complex, called the bar complex of $G$.

\textbf{Proposition. Resolution via Bar Complex.} \ We have a free resolution
\[
  \cdots \longrightarrow B_2 \longrightarrow B_1 \longrightarrow B_0 \stackrel{\varepsilon}{\longrightarrow} \mathbb{Z} \longrightarrow 0
\]
of the trivial left $\mathbb{Z}G$-module $\mathbb{Z}$.

\textbf{Proposition. Computing Group Homology.} \ Let $G$ be a group, $A$ a $G$-module and $B_\bullet$ the bar resolution of $G$. Then
\[
  H_n(G,A) ~\cong~H_n(B_\bullet \otimes_{\mathbb{Z}G} A),\qquad H^n(G,A) ~\cong~ H^n(\text{Hom}_{\mathbb{Z}G}(B_\bullet, A)).
\]

\textbf{Proposition. Interpreting $H^1(G,A)$.} \ Let $G$ be a group and $A$ a $G$-module. A map $\varphi: G \to A$ such that
\[
  \varphi(1) = 0,\qquad \varphi(gh) = \varphi(g) + g\cdot \varphi(h)
\]
for all $g,h \in G$ is called crossed homomorphism. We called crossed homomorphisms of the form $g \mapsto g\cdot a - a$ for some $a \in A$ principal. We have
\[
  H_1(G,A) ~\cong~ \frac{\{ \text{crossed homomorphisms}~G \to A \}}{\{\text{principal crossed homomorphisms}~G\to A\}}.
\]
If $A$ is the trivial $G$-module then we recover group homomorphisms and then principal crossed homomorphisms are the zero map. Thus in this case $H_1(G,A) \cong\text{Hom}_{\text{Grp}}(G,A)$.

\textbf{Proposition. (Split) Group Extensions.} \ Let $G$ be a fixed group and $A$ an abelian group. We define an extension of $G$ by $A$ to be a short exact sequence of groups
\[
  0 \longrightarrow A \stackrel{i}{\longrightarrow} E \stackrel{\pi}{\longrightarrow} G \longrightarrow 1
\]
where $\pi$ is surjective and $A \cong \text{ker}(\pi) \lhd E$ is a normal subgroup. We call such a sequence split if $\pi$ admits a section $s: G \to E$ such that $\pi \circ s = 1_G$. Note that this is not equivalent to the existence of a retract of $i$.

Given such a short exact sequence, we can give $A$ a $G$-module structure by 
\[
  g\cdot a ~=~ i^{-1}(e~i(a)~e^{-1})
\]
for any $e \in \pi^{-1}(g)$. This does not depend on the choice of preimage. In the context of $G$-modules we define an extension of $G$ by $A$ to be a short exact sequence
\[
  0 \longrightarrow A \stackrel{}{\longrightarrow} E \stackrel{}{\longrightarrow} G \longrightarrow 1
\]
such that the given $G$-module structure agrees with the one induced by the short exact sequence.

\textbf{Proposition. Semi-Direct Product and Extensions.} \ Let $G$ be a fixed group and $A$ be a $G$-module. We define $A \rtimes G$ to be a group with
\[
  A \times G,\qquad (a,g)\cdot(b,h) = (a + g\cdot h, gh).
\]
We have a split group extension of $G$ by $A$
\[
  0 \longrightarrow A \stackrel{i}{\longrightarrow} A \rtimes G \stackrel{\pi}{\longrightarrow} G \longrightarrow 1
\]
with $i(a) = (a,1)$, $\pi(a,g) = g$ and a section of $\pi$ given by $s(g) = (0,g)$. Note that this short exact sequence depends on the $G$-module structure of $A$.

\textbf{Proposition. Extensions and Equivalences.} \ An equivalence of group extensions is an isomorphism $E \cong E'$ such that the diagram commutes. Fix an extension $\eta$. We have a bijection of sets
\[
  \{ \text{equivalences}~\eta\cong \eta_A \} ~\stackrel{1:1}{\longleftrightarrow}~\{ \text{splittings}~s~\text{of}~\eta \} ~\stackrel{1:1}{\longleftrightarrow}~\text{Aut}(\eta_A).
\]

\textbf{Note.} \ For abelian groups $A$ and $B$ and
\[
  \eta:\qquad 0 \longrightarrow A \longrightarrow A \oplus B \longrightarrow B \longrightarrow 0
\]
we have $\text{Aut}(\eta) \cong \text{Hom}_{\text{Ab}}(B,A)$ (cf. matrices with a homomorphism $\varphi: B \to A$ in the upper right corner).\\
The above proposition for group extensios is analogous. Here isomorphisms between $A \rtimes G \to A \rtimes G$ are given by maps $\sigma_\varphi(a,g) = (a+g\cdot \varphi(a),g)$ for a crossed homomorphism $\varphi: G \to A$. Here we get
\[
  \text{Aut}(\eta_A) ~\cong~\{ \text{crossed homomorphisms}~G \to A \}
\]
and we recover $H^1(G,A)$ to be group extensions up to the ones which are trivial, namely those for principal crossed homomorphisms.
