\section{Tensor and Tor}

\textbf{Definition. $R$-Balanced Map.} \ Let $M \in \textsc{Mod-R}$ and $N \in \textsc{R-Mod}$ and $A \in \textsc{Ab}$. A map $\beta: M \times N \to A$ is called $R$-balanced if
\begin{enumerate}
  \item $\beta(a+b,x) = \beta(a,x) + \beta(b,x)$,
  \item $\beta(a,x+y) = \beta(a,x) + \beta(a,y)$,
  \item $\beta(ar,x) = \beta(a,rx)$.
\end{enumerate}

\textbf{Definition. Tensor Product.} \ Let $M \in \textsc{Mod-R}$ and $N \in \textsc{R-Mod}$. The tensor product of $M$ and $N$ over $R$ is an abelian group denoted $M \otimes_R N$ together with an $R$-balanced map denote $-\otimes-: M \times N \to M \otimes_R N$ such that for any $R$-balanced map $\beta: M \times N \to A$ to some abelian group $A$ there exists a unique group homomorphism $\varphi: M \otimes_R N \to A$ such that the following diagram commutes:
\[
  \begin{tikzcd}[column sep=2.5em, row sep=3.5em]
    M \times N \arrow{rr}{-\otimes_R -} \arrow[swap]{dr}{\beta} & & M \otimes_R N \arrow[dashed]{dl}{\varphi} \\ & A &
  \end{tikzcd}
  \]

  \textbf{Proposition. The Tensor Product Exists.} \ For any right $R$-module $M$ and left $R$-module $N$ the tensor product exists.

  \textbf{Proof.} \ We write $\mathbb{Z}[M \times N]$ for the free abelian group that is generated by all elements in $M \times N$. Let $U \subset \mathbb{Z}[M \times N]$ be the subgroup generated by elements of the form
  \begin{enumerate}
    \item $(m+m',n) - (m,n) - (m',n)$,
    \item $(m,n+n') - (m,n) - (m,n')$,
    \item $(mr,n) - (m, rn)$
  \end{enumerate}
  for all $m,m' \in M$, $n,n' \in N$ and $r\in R$. Then we define $M \otimes_R N := \mathbb{Z}[M\times N]/U$ and the composition
  \[
    - \otimes_R - : ~~M\times N ~\stackrel{i}{\longrightarrow}~ \mathbb{Z}[M \times N] ~\stackrel{p}{\longrightarrow}~ \mathbb{Z}[M \times N]/U
  \]
  is $R$-balanced. Furthermore, let $A$ be an abelian group. Then
  \[
    \big\{ \text{arbitrary maps}~M\times N \to A \big\} ~~\stackrel{1:1}{\longleftrightarrow}~~ \big\{ \mathbb{Z}-\text{linear maps}~\mathbb{Z}[M\times N] \big\}
  \]
  and specifying a $R$-balanced map $\beta: M\times N \to A$ is equivalent to specifying a $\mathbb{Z}$-linear map $\varphi: \mathbb{Z}[M\times N] \to A$ such that $\varphi|_U = 0$. But specifying such a map $\varphi$ is the same as specifying the map on the quotient $M \otimes_R N$.

\textbf{Note.} \ We can specify an element of the tensor product on the equivlance classes (with respect to the quotient by $U$) of basis elements $M\times N$ of $\mathbb{Z}[M \times N]$.

\textbf{Note.} \ Let $M \in R$-\textsc{Mod}. Then $R \otimes_R M \cong M$. Similarly for right $R$-modules.\\
\textbf{Note.} \ The tensor product is associative.\\
\textbf{Note.} \ If $R$ is commutative, then the tensor product is commutative.

  \textbf{Proposition. The Tensor Product is a Bifunctor.} \ Let $M \in \textsc{Mod}$-$R$ and $N \in R-\textsc{Mod}$ be module. Then the tensor product
  \[
    - \otimes_R -: \textsc{Mod}\text{-}R ~\times~ R\text{-}\textsc{Mod} ~\to~ \textsc{Ab}
  \]
  is a bifunctor.

\textbf{Note.} \ The tensor product is thus a functor in both variables. For a fixed $M$ we have a functor $N \to M \times N$ given by $(M, -)$. We then define
\[
  (M \otimes_R -) ~:=~ (-\otimes_R -) \circ (M,-):~R\text{-}\textsc{Mod} \to \textsc{Ab}.
\]
Similar for the left argument.

\textbf{Proposition. Tensor as R-Module.} \ Let $R$ be commutative and $M,N$ be $R$-modules. Then the tensor product (as a functor) lands in $R$-\textsc{Mod}.

\textbf{Proposition. The Tensor-Hom Adjunction.} \ Let $R$ be a commutative ring and $M$ be a $R$-module. Then
\[
  - \otimes_R M: \quad \dashv \quad \text{Hom}_R(M, -)
\]
is an adjunction $R\textsc{-Mod} \longleftrightarrow R\textsc{-Mod}$.

\textbf{Proof.} \ For fixed $M$ we need to find a natural isomorphism
\[
  \text{Hom}_R(L \otimes_R N, M) ~\cong~ \text{Hom}_R\big(L, \text{Hom}_R(M,N)\big).
\]
But this is a special case of the currying adjunction. Note that specifying a map on the left hand side is the same as specifying an $R$-balanced and $R$-linear map $L \times N \to M$ by the universal property of the tensor product.\\
On the other hand, the right-hand side are $R$-linear maps $L\to (M \to N)$ where $(M \to N)$ must be $R$-linear as well.\\
This is in 1:1 correspondence, thus a bijection of sets. Naturality still needs checking.

\textbf{Proposition. The Tensor Product is Right Exact.} \ Let $R$ be a commutative ring. The tensor product
\[
  (- \otimes_R -): R\textsc{-Mod} \times R\textsc{-Mod}\to R\textsc{-Mod}
\]
is right exact in both variables.

\textbf{Proof.} \ Right exactness of the first variable follows from the tensor-hom-adjunction. Since the tensor product `commutes' for commutative rings, i.e. $M \otimes_R N = N \otimes_R M$ right exactness of the second variable follows.\\
For non-commutative bimodules right exactness of the second variable can be shown by passing to the opposite ring.

\textbf{Definition. Tor.} \ Let $R$ be a ring and $N \in R$-\textsc{Mod} be fixed. We define
\[
  \text{Tor}_n^R(-, N): \textsc{Mod-}R \to \textsc{Ab},\qquad \text{Tor}_n^R(-,N) = L_n^R(- \otimes_R N)
\]
to be the $n$-th left derived functor of $(- \otimes_R N)$. If $R$ is commutative both the tensor product and thus also $\text{Tor}_N^R$ land in $R$-\textsc{Mod}.

\textbf{Definition. Flat Module.} \ A right $R$-module is called flat if $(M \times_R -)$ is exact. Similarly for left $R$-modules.\\
Since $(M \otimes_R -)$ is right exact $M$ is flat if and only if it preserves kernels (or equivalently monos (?)).

\textbf{Proposition. Projectives are Flat.} \ Projective left and right $R$-modules are flat.

\textbf{Proposition. Tor is Symmetric.} \ Let $M \in \textsc{Mod-}R$ and $N \in R\textsc{-Mod}$ be fixed. We have a natural isomorphism
\[
  L_n^R(- \otimes_R N) ~\cong~ L_n^R(M \otimes_R -).
\]

\textbf{Definition. Acyclic.} \ Let $F: \mathcal{A} \to \mathcal{B}$ be left (right) exact. An object $A \in \mathcal{A}$ is called $F$-acyclic if $R^nF(A) = 0$ ($L_nF(A) = 0$) for all $n \geq 0$.

\textbf{Note.} \ Flat $R$-modules are $(-\otimes -)$-acyclic, projective objects are $\text{Hom}(-,B)$ acyclic and injective objects $\text{Hom}(A,-)$-acyclic.

\textbf{Note.} \ Projective and injective objects are the desirable objects for any right and left exact functor, however for a specific functor there might be more, namely the acyclic objects for that functor. These acyclics equally well produce the derived functor and may be easier to find and handle.

\textbf{Proposition. Derived Functor via Acylic Resolution.} \ Let $\mathcal{A}$ be an abelian category with enough projectives and $F: \mathcal{A} \to \mathcal{B}$ a right exact functor. Let
\[
  X_\bullet \twoheadrightarrow A
\]
be a acyclic resolution of any $A \in \mathcal{A}$. Then
\[
  L_nF(A) ~\cong~H_n\big(F(X_\bullet)\big).
\]
We obtain the derived functor by this acyclic resolution.
